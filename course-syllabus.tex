\documentclass[a4paper]{article}

\usepackage{hyperref}
\usepackage{geometry}
\usepackage{multirow}
\usepackage{tabularx}
\usepackage{float}
\usepackage{makecell}
  
\newcolumntype{C}{>{\centering\arraybackslash}X} % custom X column with horizontal centering

% \usepackage{soul}
% % \newcommand{\new}[1]{\hl{#1}}
\newcommand{\new}[1]{{#1}}

\begin{document}

\begin{center}
  {\bfseries {\huge CS 491 Kaavish I}\\\bigskip
    {\large 3+0 credits, Fall 2022, Habib University}}\\\medskip

\textit{``Education is not the learning of facts, but the training of the mind to think.''\\
  -- Albert Einstein}
\end{center}

\noindent
\begin{tabularx}{1.0\linewidth}{lX}
  Instructors: & TBD\\
  Student hours: & TBD\\
  Class time: & Fridays 1130h--1245h\\
  Course sites: & LMS (Canvas), Yammer, Live Syllabus\\
  Course Prerequisites: & CS 201, CS 353, being a CS senior\\
  Content Area: & This course is required for the successful completion of a major in CS.
\end{tabularx}
\medskip

\section{Rationale}
\new{This course, CS 491 Kaavish I, is the first half of Kaavish which is} a year-long experience for Computer Science seniors that requires them to solve a sufficiently challenging real-world problem. One of the key objectives of Kaavish is to enable students to relate theoretical concepts and practical applications. Kaavish is a reflection of your four years of intellectual journey that prepares you to integrate your knowledge and skills to make an effective contribution to the society.

\section{Course Aims and Outcomes}
\subsection{Aims}
Kaavish condenses your 4 years with the CS program at Habib University into a single course in the form of a final year project. The aim of the course is for you to research, propose, solve, and present a practical and meaningful problem using your skills and concepts in an allocated time frame. Broadly speaking, Kaavish aims for you by the end of this course to be able to
\begin{itemize}
\item given a particular situation, identify ways to improve the associated work flows
\item brainstorm different solutions for a given problem
\item assess different solutions in order to identify the most feasible one within a given set of constraints
\item successfully plan the execution of a large project while complying with scope, time, and resource constraints
\item design, implement, and test a working system/model as per state of the art practices
\item apply agile methodologies to execute a project
\item exercise leadership and team building skills
\item work responsibly, independently, and as part of a team
\item communicate your idea and solution effectively to a large audience via different mediums--presentations, reports, posters and so on.
\end{itemize}

\new{Specifically, Kaavish I requires you to complete the necessary research and experimentation and build a viable prototype of your eventual software.}

\subsection{Course Learning Outcomes (CLOs)}

\new{Course Learning Outcomes (CLOs) describe your knowledge, skills, and abilities on successful completion of a course.}

\new{By successfully completing Kaavish, you will achieve the following outcomes at the indicated} \href{https://educerecentre.com/what-are-the-three-domains-of-blooms-taxonomy/}{learning levels}.

\noindent
\begin{tabularx}{\textwidth}{|l|X|l|}
\hline
CLO 1 & Analyse problem complexity and find an optimal and practical solution & COG-4 \\\hline
CLO 2 & Design a system which is robust and scalable & COG-4\\\hline
CLO 3 & Write production ready code using appropriate programming languages & COG-4\\\hline
CLO 4 & Deploy a system in real world and create a suitable demo & COG-3\\\hline
CLO 5 & Perform related research and learn new tools as appropriate & COG-3\\\hline
CLO 6 & Describe the impact of your work on the society & COG-3\\\hline
CLO 7 & Work in teams, take ownership of sub-systems, and provide constructive feedback to other team members & COG-3\\\hline
\end{tabularx}

\subsection{Program Learning Outcomes (PLOs) }

\new{Program Learning Outcomes (PLOs) describe your knowledge, skills, and abilities at the time of graduation. Shown below is the level of emphasis (1:High; 2:Medium; 3:Low) that each of the CLOs places on the PLOs of the Computer Science program.}

\noindent\small
\begin{tabularx}{\linewidth}{|l|X||*{7}{c|}}
  \cline{3-9}
  \multicolumn{2}{c}{} & \multicolumn{7}{|c|}{\textbf{CLO}} \\\hline
  \multicolumn{2}{|c||}{\textbf{PLO}} & \textbf{1} & \textbf{2} & \textbf{3} & \textbf{4} & \textbf{5} & \textbf{6} & \textbf{7} \\\hline
 1 & Analysis: analyze a given situation and reduce it to one or more problems that can be solved via computer intervention. & 1 &&&&&&\\\hline
2 & Design: design one or more computer-based solutions of a given problem and select the solution that is best under the circumstances. &&1&&&&& \\\hline
 3 & Programming: program a given solution in a variety of programming languages belonging to different paradigm. &&&1&&&& \\\hline
 4 & Implementation: design and implement software systems of varying complexity. &&&&1&&& \\\hline
 5 & Tools: work with the latest tools that support development, e.g., IDE's, version control systems, debuggers, profilers, and continuous build systems. &&&&1&&& \\\hline
 6 & Self-learning: research, learn, and apply requirements needed to implement a solution for a given high level problem. &&&&&1&& \\\hline
 7 & Ethics and Awareness: foresee both impact and possible ramifications of computing practices &&&&&&1& \\\hline
 8 & Communication and Teamwork: work effectively in inter-disciplinary teams. &  &  &  &  &  & &1\\\hline
\end{tabularx}
\normalsize

\section{Format and Procedures}
\label{sec:format}

\new{Kaavish is a year-long team project divided into 2 courses--CS 491 Kaavish I and CS 492 Kaavish II. The main elements of your Kaavish are the problem that your Kaavish solves and the solution that your team ideates, designs, develops, builds, and deploys employing current best practices.}

\new{The main milestones in Kaavish I are team formation, project proposal and defense, and final evaluation.}

\new{The first few weeks of Kaavish I are devoted to forming your Kaavish team. You will work within this team for the remainder of the year on a project that you will ideate on with your team. At the proposal defense, all teams present their proposed projects. You then work toward a prototype which you present and defend in the final evalaution at the end of the semester. Along the way you attend sessions designed to facilitate your work in various ways.}


\subsection{Team}

Kaavish is a team project. It is your responsibility to assemble like minded peers around your project idea in order to form a team. Every Kaavish team must comprise three to five (inclusive) members, at least two of which must be eligible CS majors. All the CS majors in the team must be enrolled together in CS 491 Kaavish I. 

You are encouraged to prepare your team before the start of Fall 2022. You officially announce your team in your Kaavish proposal form. A team is finalized once its proposal is accepted. Thereafter, you work with the same team members for the remaining duration of your Kaavish.

The choice of team members is therefore crucial to the success of your Kaavish. You have had various opportunities to work with your CS peers over the last 3 years with the CS program. A good team mate is one whom you have found to be dependable and reliable. They share your interest in your proposed Kaavish project and you have found their work ethic to suit yours, either as a complement or because it is similar. Over time you must have learned that a good work partner is not necessarily a good social partner and vice versa. You are expected to have developed good team dynamics through your time in the CS program so far.

Working in a team, each member will receive an individual score based on their contribution to the team. This dynamic between individual and team work is deliberate and each team ends up developing its own norms and boundaries.

\subsection{Enrollment}

Change of section is allowed in this course from the first week till the eighth week of the semester. This facilitates the formation of effective Kaavish teams whose members may originally be enrolled in separate sections.

\section{Assessments and Grading}

The table below shows the assessments that you will attempt in this course, along with the weightage of each for your final score. The university's standard mapping for scores to grades is also shown.

\begin{center}
\begin{tabular}{|l|l|}
  \hline
  \textbf{Assessment} & \textbf{Weightage} \\\hline\hline
  Project proposal & 10\% \\\hline
  Weekly progress & 20\% \\\hline
  Deliverables & 30\% \\\hline
  Final evaluation & 30\% \\\hline
  Professional visits & 10\% \\\hline
\end{tabular}
$\quad$
  \begin{tabular}{|*3{c|}}
\hline
\textbf{Grade} & \textbf{Points} & \textbf{Percentage}\\\hline\hline
A+ & 4.00 & [95-100] \\\hline
A & 4.00 & [90-95) \\\hline
A- & 3.67 & [85-90) \\\hline
B+ & 3.33 & [80-85) \\\hline
B & 3.00 & [75-80) \\\hline
B- & 2.67 & [70-75) \\\hline
C+ & 2.33 & [67-70) \\\hline
C & 2.00 & [63-67) \\\hline
C- & 1.67 & [60-63) \\\hline
F & 0.00 & [0-60) \\\hline
  \end{tabular}
\end{center}

\new{Your team's project proposal is assessed in the proposal defense on the suitability of the proposed project. The proposal is expected to demonstrate effort, thought, research, and planning on a carefully chosen problem at an intellectual standard befitting a team of CS seniors at Habib University.}

\new{Once your team's proposed project is approved, the team begins work on it and provide weekly updates on your progress.}

\new{Various deliverables at specific times of the semester will measure your team's progress. These include a project plan, Software Requirement Specification (SRS) and Software Design Specification (SDS) documents, and a literature review.}

\new{The final evalaution assesses the extent to which your team has completed its research and developed a viable prototype of the eventual software.}

\new{Your instructor will regularly share external events of interest to the Kaavish process. You are required to attend a certain number of these.}

\section{Attendance Policy}

\new{You are expected to attend the weekly sessions. To account for unforeseen circumstances a relaxatin of 15\% is allowed.}

\section{Accommodations for Students with Disabilities}
In compliance with the Habib University policy and equal access laws, Kaavish is available to discuss appropriate academic accommodations that may be required for student with disabilities. Requests for academic accommodations are to be made during the first two weeks of the semester, except for unusual circumstances, so arrangements can be made. Students are encouraged to register with the Office of Academic Performance to verify their eligibility for appropriate accommodations.

\section{Inclusivity Statement}
We understand that our members represent a rich variety of backgrounds and perspectives. Habib University is committed to providing an atmosphere for learning that respects diversity. While working together to build this community we ask all members to:
\begin{itemize}
	\item share their unique experiences, values and beliefs
	\item be open to the views of others 
	\item honor the uniqueness of their colleagues
	\item appreciate the opportunity that we have to learn from each other in this community
	\item value each other’s opinions and communicate in a respectful manner
	\item keep confidential discussions that the community has of a personal (or professional) nature 
	\item use this opportunity together to discuss ways in which we can create an inclusive environment in this course and across the Habib community 
\end{itemize}

\section{Academic Integrity}

Each student in this course is expected to abide by the Habib University Student Honor Code of Academic Integrity.  Any work submitted by a student in this course for academic credit will be the student's own work. There is zero tolerance for plagiarism. Every case will be reported to the conduct office and you'll get a zero on that particular test or assignment.

\new{This is a team-based course. Collaboration with your team-mates, once assigned, is allowed in all team-based deliverables and assessments.}

Scholastic dishonesty shall be considered a serious violation of these rules and regulations and is subject to strict disciplinary action as prescribed by Habib University regulations and policies. Scholastic dishonesty includes, but is not limited to, cheating on exams, plagiarism on assignments, and collusion. 

\noindent{\bf PLAGIARISM}: Plagiarism is the act of taking the work created by another person or entity and presenting it as one’s own for the purpose of personal gain or of obtaining academic credit. As per University policy, plagiarism includes the submission of or incorporation of the work of others without acknowledging its provenance or giving due credit according to established academic practices. This includes the submission of material that has been appropriated, bought, received as a gift, downloaded, or obtained by any other means. Students must not, unless they have been granted permission from all faculty members concerned, submit the same assignment or project for academic credit for different courses. 

\noindent{\bf CHEATING}: The term cheating shall refer to the use of or obtaining of unauthorized information in order to obtain personal benefit or academic credit. 

\noindent{\bf COLLUSION}: Collusion is the act of providing unauthorized assistance to one or more person or of not taking the appropriate precautions against doing so. 
All violations of academic integrity will also be immediately reported to the Student Conduct Office.  

You are encouraged to study together and to discuss information and concepts covered in lecture and the sections with other students. You can give "consulting" help to or receive "consulting" help from such students. However, this permissible cooperation should never involve one student having possession of a copy of all or part of work done by someone else, in the form of an e-mail, an e-mail attachment file, a diskette, or a hard copy. 

Should copying occur, the student who copied work from another student and the student who gave material to be copied will both be in violation of the Student Code of Conduct. 

During examinations, you must do your own work. Talking or discussion is not permitted during the examinations, nor may you compare papers, copy from others, or collaborate in any way. Any collaborative behavior during the examinations will result in failure of the exam, and may lead to failure of the course and University disciplinary action.

Penalty for violation of this Code can also be extended to include failure of the course and University disciplinary action. 

\newpage
\section{Tentative Course Schedule }

{\it May change to accommodate student needs}

\begin{center}
  \begin{tabular}{|r|*{3}{l|}}
    \hline
    \textbf{Week} & \textbf{Topic} & \textbf{Activity} & \makecell{\textbf{Assessments/}\\\textbf{Deliverables}} \\\hline\hline
    1. & \makecell[l]{Elements of team work;\\ Your Kaavish proposal} & Consultation hour & \\\hline
    2. & & \makecell[l]{Field trip;\\Consultation hour} & \\\hline
    3. & Project brain-storming & Consultation hour & \\\hline
    4. & User-centered design & Consultation hour & \\\hline
    5. &  & Consultation hour & Project proposal defense \\\hline
    6. & \makecell[l]{Elements of an SRS;\\Gantt charts} & \makecell[l]{Stand-up meeting\\Consultation hour} & \\\hline
    7. & \makecell[l]{Review of software engineering\\ practices} & \makecell[l]{Stand-up meeting\\Consultation hour} & \\\hline
    8. &  & \makecell[l]{Progress update\\ presentations;\\Consultation hour} & \\\hline
    9. & Building a prototype & \makecell[l]{Stand-up meeting\\Consultation hour} & \\\hline
    10. & Your evaluation presentation & \makecell[l]{Stand-up meeting\\Consultation hour} & Project-plan and SRS due\\\hline
    11. & & \makecell[l]{Progress update\\ presentations;\\Consultation hour} & \\\hline
    12. & \makecell[l]{Elements of an SDS;\\Writing a literature review} & \makecell[l]{Stand-up meeting\\Consultation hour} & \\\hline
    13. & Proejct to product (guest talk) & \makecell[l]{Stand-up meeting\\Consultation hour} meeting & \\\hline
    14. & & \makecell[l]{Progress update\\ presentations;\\Consultation hour} & SDS due\\\hline
    15. & Prototype review & \makecell[l]{Stand-up meeting\\Consultation hour} & \\\hline
     &  &  & Final Evaluation \\\hline
  \end{tabular}
\end{center}

\end{document}

%%% Local Variables:
%%% mode: latex
%%% TeX-master: t
%%% End:
